\documentclass[10pt,a5paper,computeautoilg,pdfinfo=on]{unistyle}

\usepackage{url}
\usepackage{ptext}
\usepackage[colorlinks=true]{hyperref}

%\baselineskip=.61cm

\begin{document}

\SyllabusMakeTitle{گروه علوم کامپیوتر (ترم بهار، \LR{1397-98})}{مبانی نظريه محاسبات}{1316103}

\tbox{آموزشیار دوره}
\ptext[1]

\tbox{محل و زمان کلاس}
‌دوشنبه \clock[13:30] تا \clock[15:00]، کلاس ۱۳۵. سه‌شنبه(زوج) \clock[13:30] تا \clock[15]، کلاس ۱۹۲.

\tbox{منبع درس}
\begin{latin}
%\begin{itemize}
%\item
J.E. Hopcroft, R. Motwani, J.D. Ullman, ‍‍‍‍``\textsf{Introduction to Automata Theory, Languages, and Computation}'', 2nd edition‬‬.\footnotemark
%\end{itemize}
\end{latin}

\footnotetext{\rl{نسخه الکترونیکی این کتاب از شبکه اینترنت قابل دریافت است.}}

\tbox{مراجع دیگر}
\begin{latin}
\begin{itemize}
%\setlength{\itemsep}{-4mm}
	\item John C. Martin, ``\textsf{Introduction to languages and the theory of computation}", 3rd edition.
	\item Peter Linz, ``\textsf{An introduction to formal languages and automata}", 4th edition.
	\item \url{http://jflap.org}
\end{itemize}
\end{latin}

\tbox{پیشنیاز}
%‫ریاضیات گسسته ۹۰۴۵۴۲۲ و اصول سیستمهای کامپیوتری ۱۰۴۵۴۲۲‬
مبانی ‌علوم‌ریاضی و تا حدّی ریاضیات~گسسته و آشنایی با برنامه‌نویسی مقدماتی

\tbox{کمک‌یار آموزشی}
در حال حاضر هیچ کمک‌یار آموزشی تعیین نشده است لکن در صورت همکاری یکی از دانشجویان سال بالاتر به عنوان حل تمرین،
اطلاع‌رسانی خواهید شد.

\tbox{محتوای درس}
‫مقدم‌های بر زبان‌های رسمی و گرامرها. ماشین‌های متناهی قطعی و غیرقطعی. زبان‌های منظم. عبارات منظم. محدودیت زبان‌ها.
گرامرهای مستقل‬ ‫از متن. زبان‌های مستقل از متن. ماشین‌های پشته‌ای. سلسله مراتب چامسکی. ماشین تورینگ.‬%
\footnote{در صورت نقصان این وظیفه شماست که مابقی مطالب را خودتان مطالعه نمایید. برای جزئیات بیشتر می‌توانید به سیلابس درس که از طریق وبگاه دانشگاه قابل دسترسی است مراجعه نمایید.}

%  \newpage

\tbox{سختی درس}
با توجه به اینکه جنبهٔ نظری این درس بر جنبهٔ عملی آن در طول ارائه، غلبه دارد لذا نیاز چندانی به برنامه‌نویسی نخواهید داشت و
 به نسبت زیادی، «مبانی نظریه محاسبات» درسی ریاضی محور است و آشنایی با ریاضیات~گسسته ضروری است
لذا همانند سایر دروس پایهٔ ریاضیات تنها راه یادگیری بهتر، تمرین و تمرین و تمرین است.
و البته که شما در این درس بیشتر خواهید اندیشید!

\tbox{سیاست نمره‌گذاری\protect\footnotemark}
\footnotetext{احتمال نقصان یا زیادت در هر کدام از گزینه‌ها وجود دارد که طبعاً به اطلاع خواهد رسید.}
کتاب به دو بخش تقسیم خواهد شد که نیمی از آن در میان‌ترم و مابقی در پایان‌ترم ارزیابی خواهد شد.
\begin{RTLcases}\begin{tabular}{rl}
میان‌ترم & $50$\%   \\
پایان ترم & $50$\%
\end{tabular}\end{RTLcases}

\tbox{بایدها و نبایدها}
\begin{multicols}{2}
\begin{enumerate}
\item به موقع سر کلاس باشید.
\item پس از شروع کلاس لطفاً دیگر تشریف نیاورید.
\item قبل از ورود به کلاس تلفن همراه خود را خاموش نمایید.
%\item { به ازای هر جلسه تاخیر ۰/۱۲۵ از نمره کل کسر خواهد شد، و هر دو جلسه تاخیر نیز یک جلسه غیبت محسوب می‌گردد. }
%\item { به ازای هر جلسه غیبت ۰/۲۵ از نمره کل کسر خواهد شد، و پس از ۴ جلسه غیبت از کلاس حذف خواهید شد.}
\item درباره نمرهٔ نهایی بحث نفرمایید، نمره تغییر نخواهد کرد.
\item نمرات حاشیه‌ای انتظار انتقال به ۱۰ را نداشته باشند.
\item هیچ نمره‌ای از یک ترم به ترم دیگر منتقل نخواهد شد.
\item تمام تمارین و تکالیف را انجام دهید حتی اگر پیش از این یکبار دیگر درس را گرفته‌ باشید --منتظر سررسید تحویل تمارین نباشید تا آن را شروع نمایید!--.
\item تاریخ تحویل تمارین و پروژه‌ها قابل تمدید نمی‌باشد.
\item در صورت شناسایی تمارین یا پروژه‌های کپی شده به هر دو طرف --نویسنده اصلی و فرد کپی گیرنده-- نمره \underline{صفر}   تعلق خواهد گرفت.
\item در هفته منتهی به امتحان هیچگونه سوالی پاسخ داده نخواهد شد.
\item برای ارسال ایمیل علاوه بر انتخاب موضوع درخور \underline{ حتماً در موضوع نامه عبارت \lr{[FLS2019]}} را نیز قرار دهید.
\item از نگارش ایمیل با زبانی غیر از فارسی یا انگلیسی بپرهیزید --منظور زبان موهوم فینگلیش است--.
\end{enumerate}
\end{multicols}


\leftline{{\tiny نسخه 0}}
\vfill\kalamehakim
\xepersianproof
\end{document}
