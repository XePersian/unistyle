\documentclass[a4paper,12pt]{unistyle}

\usepackage{url}
\usepackage{amsmath}

\unidetails{
عنوان درس=مبانی کامپیوتر و برنامه‌سازی,
مقطع و رشته=ک‍‍ارش‍ن‍‍اس‍‍ى‌ ‌ع‍ل‍وم‌‌ک‍‍ام‍پ‍ی‍وت‍ر,
روز و تاریخ=دوشنبه \LR{1398/11/01},
ترم=1,
سال تحصیلی=98--99,
مدت امتحان=180,
نوع آزمون=پایان‌ترم,
لوگوی دانشگاه=logo,
}


\begin{document}
\MakeTitle
\begin{enumerate}

    \item
    پرسش‌های زیر را در \textbf{همین برگه} پاسخی کوتاه دهید. \grade{2}
    \begin{enumerate}[itemtwocol]
        \parskip=20pt
        \item تنها با یک دستور و  در یک خط سه متغیر تعریف نموده و مقدار اولیه آنها را صفر قرار دهید.
        \item \LRE{\texttt{"\{0\} * \{1\} is not \{0\}".format("*", 5)}}؟
    \end{enumerate}


    \item سوال تستی چهارستونی ستونی       \grade{4}
        \begin{multiplechoices}
            \choice گزینه اول
            \choice گزینه دوم
            \choice گزینه سوم
            \choice گزینه چهارم
            \choice گزینه پنجم
            \choice گزینه ششم
            \choice گزینه هفتم
            \choice گزینه هشتم
            \choice گزینه نهم
            \choice گزینه دهم
            \choice گزینه یازدهم
            \choice گزینه دوازدهم
        \end{multiplechoices}

        \item سوال تستی دوستونی ستونی
        \begin{multiplechoices}
            \choice گزینه اول  گزینه اول گزینه اول
            \choice گزینه دوم  گزینه دوم گزینه دوم
            \choice گزینه سوم  گزینه سوم گزینه سوم
            \choice گزینه چهارم گزینه چهارم گزینه چهارم
        \end{multiplechoices}

    \item سوال تستی تک‌ستونی ستونی
        \begin{multiplechoices}
            \choice گزینه اول گزینه اول گزینه اول گزینه اول گزینه اول
            \choice گزینه دوم گزینه دوم گزینه دوم گزینه دوم گزینه دوم
            \choice گزینه سوم گزینه سوم گزینه سوم گزینه سوم گزینه سوم
            \choice گزینه چهارم گزینه چهارم گزینه چهارم گزینه چهارم
        \end{multiplechoices}
\end{enumerate}

\makeresponseform{40}

\sign
\vfill
\kalamehakim
\xepersianproof

\end{document}
